\documentclass[10pt]{IEEEtran}
\usepackage{hyperref}
\usepackage[latin1]{inputenc}

\title{Active Queue Management}
\author{Thomas Fischer, Dominik Billing}

\begin{document}
\maketitle

\section{Einf�hrung und Motivation}
\subsection{Problem}
Das Internet hat mittlerweile eine Gr��e erreicht, bei der man sich nicht mehr komplett auf End-To-End Staukontrolle verlassen kann \cite{Floyd1997}.

\subsection{Problemstellung}
Die mittlere Pufferauslastung soll gering gehalten werden \cite{Le2003}.

\section{Staukontrolle in Netzwerken}

\subsection{Vorschl�ge zur Staukontrolle}
Vorschl�ge zur Gew�hrleistung und Verbesserung der Internetperformance \cite{Braden1998}.

\subsection{congestion collapse}
Congestion collapse und wie man es dazu f�hrt \cite{Nagle1984}.

\subsection{Warum ist das Problem so schwer zu identifizieren}
Was sind die Gr�nde f�r das Problem \cite{Jain1990}?

\subsection{Mechanismen zur Staukontrolle in ATM Netzwerken}
Auswahlkriterien zwischen den beiden Ans�tzen rate-based und credit-based \cite{Jain1996}.

\subsection{Standard TCP Verhalten bei Staus}
Warum ist es keine gute Idee TCP die Staukontrolle selbst zu machen \cite{Morris1997}? Gleichbehandlung aller Datenstr�me \cite{Suter1998}.

\subsection{Explicit Congestion Notification}
Vor- und Nachteile von ECN bei TCP \cite{Floyd1994}

\section{Definition und Anwendung von Active Queue Management}

Efficient Active Queue Management for Internet Routers											\url{http://www.researchgate.net/publication/2621818_Efficient_Active_Queue_Management_for_Internet_Routers}

Router Mechanisms to Support End-to-End Congestion Control										\url{http://citeseerx.ist.psu.edu/viewdoc/summary?doi=10.1.1.39.7772}

Sizing router buffers																			\url{http://dl.acm.org/citation.cfm?id=1015499}

Stochastic Modeling and the Theory of Queues													\url{http://www.gbv.de/dms/ilmenau/toc/018830102.PDF}

Analysis and simulation of a fair queueing algorithm											\url{http://dl.acm.org/citation.cfm?id=75248}

\section{Algorithms for Active Queue Management}

\subsection{Passive Techniques}
\subsubsection{CHOKe}
CHOKe - a stateless active queue management scheme for approximating fair bandwidth allocation	\url{http://ieeexplore.ieee.org/xpl/login.jsp?tp=&arnumber=832269&url=http\%3A%2F%2Fieeexplore.ieee.or0\%2Fxpls%2Fabs_all.jsp\%3Farnumber%3D832269}
\subsubsection{ECN}
ECN-Explicit Congestion Notification															http://www.hjp.at/doc/rfc/rfc3168.html
\subsection{Random Early Detection}
RED														Random early detection gateways for congestion avoidance										\url{http://ieeexplore.ieee.org/xpl/articleDetails.jsp?arnumber=251892&navigation=1}
	RED + Vorschl�ge f�r Architektur etc.					A study of active queue management for congestion control 										\url{http://ieeexplore.ieee.org/xpl/login.jsp?tp=&arnumber=832541&url=http\%3A\%2F\%2Fieeexplore.ieee.org\%2Fxpls\%2Fabs_all.jsp\%3Farnumber\%3D832541}

RED + Vorschl�ge f�r Architektur etc.					A study of active queue management for congestion control 										\url{http://ieeexplore.ieee.org/xpl/login.jsp?tp=&arnumber=832541&url=http\%3A%2F%2Fieeexplore.ieee.org\%2Fxpls%2Fabs_all.jsp\%3Farnumber%3D832541}

\subsection{Alternatives to RED}

\subsubsection{PI Controller}

PI														On designing improved controllers for AQM routers supporting TCP flows 							\url{http://ieeexplore.ieee.org/xpl/login.jsp?tp=&arnumber=916670&url=http\%3A%2F%2Fieeexplore.ieee.org\%2Fxpls%2Fabs_all.jsp\%3Farnumber%3D916670}

Vergleich RED, ARED, PI									The effects of active queue management on web performance										\url{http://dl.acm.org/citation.cfm?id=863986}

\subsubsection{Adaptive Virutal Queue}
Adaptive virtual Queue + (Vergleich zu RED, REM, PI)	Analysis and design of an adaptive virtual queue (AVQ) algorithm for active queue management	\url{http://dl.acm.org/citation.cfm?id=383069}

\subsubsection{BLUE}
BLUE													The BLUE active queue management algorithms														\url{http://dl.acm.org/citation.cfm?id=581869}

\section{Conclusion and Future Work} 
\nocite{*}

\bibliographystyle{IEEEtran}
\bibliography{IEEEabrv,bibliothek}

\end{document}

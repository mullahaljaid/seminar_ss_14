\documentclass[10pt]{IEEEtran}
\usepackage{hyperref}
\usepackage[latin1]{inputenc}
\usepackage[ngerman]{babel}

\title{Staukontrolle durch Active Queue Management}
\author{Thomas Fischer, Dominik Billing}

\begin{document}
\maketitle

\begin{abstract}
Dieser Artikel beschreibt die Problematik im Internet, die durch den Einsatz von konventioneller Staukontrolle in Routern und die Struktur des Internets hervorgerufen wird. Die Problematik konventioneller Staukontrolle liegt darin, dass Pakete wahllos fallen gelassen werden und es auf diese Art zu gro�em Overhead kommt, der durch Flaschenh�lse im Internet noch verst�rkt wird.

Wir werden als L�sung f�r das Problem der Staukontrolle im Internet Active Queue Management herausarbeiten. Hierbei werden Pakete nicht wahllos fallen gelassen, sondern im Gegensatz alle Fl�sse gleich behandelt und nachfolgende Router mittels markierter Pakete dar�ber informiert, dass es zu Staus kommen kann. Active Queue Management ist der �berbegriff von Methoden, um die mittlere Pufferauslastung der Router m�glichst gering zu halten. Anschlie�end werden wir die Active Queue Management Methoden RED, BLUE, PI und ECN vorstellen und miteinander vergleichen.
\end{abstract}

\section{Einf�hrung und Motivation}
\begin{itemize}
\item Problem

Das Internet hat mittlerweile eine Gr��e erreicht, bei der man sich nicht mehr komplett auf Ende-zu-Ende Staukontrolle verlassen kann \cite{Floyd1997}. Sehr viele Router stellen aufgrund der sternf�rmigen Struktur des Internets Flaschenh�lse dar.
\item Problemstellung

Die mittlere Pufferauslastung soll gering gehalten werden, um E2E Staukontrolle zu erm�glich \cite{Le2003}.
\end{itemize}



\section{Staukontrolle in Netzen}
\begin{itemize}
\item Vorschl�ge zur Staukontrolle

Vorschl�ge zur Gew�hrleistung und Verbesserung der Internetperformance \cite{Braden1998}
\item congestion collapse

Congestion collapse und wie es dazu f�hrt \cite{Nagle1984}.
\item Warum ist das Problem so schwer zu identifizieren

Was sind die Gr�nde f�r das Problem \cite{Jain1990}?
\item Mechanismen zur Staukontrolle in ATM Netzwerken

Auswahlkriterien zwischen den beiden Ans�tzen rate-based und credit-based \cite{Jain1996}.
\item Standard TCP Verhalten bei Staus

Warum ist es keine gute Idee TCP die Staukontrolle selbst zu machen \cite{Morris1997}? Gleichbehandlung aller Datenstr�me \cite{Suter1998}.
\item Explicit Congestion Notification

Vor- und Nachteile von ECN bei TCP \cite{Floyd1994}
\item Router Mechanismen zur Staukontrolle

Vor- und Nachteile von normaler Staukontrolle in Routern \cite{Floyd1997}
\item �berleitung zu Active Queue Management

Active Queue Management ist eine L�sungsansatz zur Staukontrolle \cite{Graffi2007}
\end{itemize}



\section{Definition und Anwendung von Active Queue Management}
\begin{itemize}
\item Effizientes Active Queue Management in Internet Routern \cite{Suter1998a}
\item Dimensionierung von Router Puffern \cite{Appenzeller2004}
\item Stochastische Modellierung und die Theorie von Queues \cite{Wolff1998}
\item Analyse und Simulation eines gleichbehandelnden Queue Algorithmus \cite{Demers1989}
\end{itemize}



\section{Die g�ngigsten Active Queue Management Algorithmen}
\begin{itemize}
\item RED (Random Early Detection) \cite{Floyd1993} \cite{Firoiu2000}

Adaptive RED\cite{Floyd2001}
\item BLUE \cite{Feng2002}
\item ECN \cite{Ramakrishnan2001}
\item PI Controller \cite{Hollot2001}
\end{itemize}



\section{Vergleich der vorgestellten Algorithmen}
\begin{itemize}
\item Vergleich RED, ARED, PI \cite{Le2003}
\item Vergleich RED, PI \cite{Kunniyur2001}
\item Vergleich RED, BLUE, ARED, ECN, PI \cite{Graffi2007}
\end{itemize}



\section{Ausblick und andere Ans�tze}
\begin{itemize}
\item Ein wirklich optimaler Algorithmus muss noch gefunden werden \cite{Graffi2007}
\item Statt Staukontrolle andere Wege suchen (CHOKe) \cite{Pan2000}
\end{itemize}

\nocite{*}
\bibliographystyle{IEEEtran_de}
\bibliography{IEEEabrv,bibliothek}

\end{document}

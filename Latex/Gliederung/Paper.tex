\documentclass[10pt]{IEEEtran}
\usepackage{hyperref}
\usepackage[latin1]{inputenc}

\title{Active Queue Management}
\author{Thomas Fischer, Dominik Billing}

\begin{document}
\maketitle

\section{Einf�hrung und Motivation}
\subsection{Problem}
Das Internet hat mittlerweile eine Gr��e erreicht, bei der man sich nicht mehr komplett auf End-To-End Staukontrolle verlassen kann \cite{Floyd1997}.
\subsection{Problemstellung}
Die mittlere Pufferauslastung soll gering gehalten werden \cite{Le2003}.

\section{Staukontrolle in Netzwerken}
\subsection{Vorschl�ge zur Staukontrolle}
Vorschl�ge zur Gew�hrleistung und Verbesserung der Internetperformance \cite{Braden1998}.
\subsection{congestion collapse}
Congestion collapse und wie man es dazu f�hrt \cite{Nagle1984}.
\subsection{Warum ist das Problem so schwer zu identifizieren}
Was sind die Gr�nde f�r das Problem \cite{Jain1990}?
\subsection{Mechanismen zur Staukontrolle in ATM Netzwerken}
Auswahlkriterien zwischen den beiden Ans�tzen rate-based und credit-based \cite{Jain1996}.
\subsection{Standard TCP Verhalten bei Staus}
Warum ist es keine gute Idee TCP die Staukontrolle selbst zu machen \cite{Morris1997}? Gleichbehandlung aller Datenstr�me \cite{Suter1998}.
\subsection{Explicit Congestion Notification}
Vor- und Nachteile von ECN bei TCP \cite{Floyd1994}

\section{Definition und Anwendung von Active Queue Management}
\subsection{Effizientes Active Queue Management in Internet Routern}
\cite{Suter1998a}
\subsection{Router Mechanismen zur Staukontrolle}
\cite{Floyd1997}
\subsection{Dimensionierung von Router Puffern}
\cite{Appenzeller2004}
\subsection{Stochastische Modellierung und die Theorie von Queues}
\cite{Wolff1998}
\subsection{Analyse und Simulation eines gleichbehandelnden Queue Algorithmus}
\cite{Demers1989}

\section{Active Queue Management Algorithmen}
\subsection{Passive Techniques}
\subsubsection{CHOKe}
\cite{Pan2000}
\subsubsection{ECN}
\cite{Ramakrishnan2001}
\subsection{Random Early Detection}
\cite{Floyd1993}
\cite{Firoiu2000}
Adaptive RED \cite{Floyd2001}
\subsection{Alternativen zu RED}
\subsubsection{PI Controller}
\cite{Hollot2001}
\subsubsection{Vergleich RED, ARED, PI}
\cite{Le2003}
\subsubsection{REM}
\cite{Athuraliya2001}
\subsubsection{Adaptive Virutal Queue + Vergleich zu RED, REM, PI}
\cite{Kunniyur2001}
\subsubsection{BLUE}
\cite{Feng2002}
\subsubsection{Vergleich RED, BLUE, ARED, PI, ECN, REM}
\cite{Graffi2007}

\section{Ausblick und zuk�nftige Arbeiten}
Ein wirklich optimaler Algorithmus muss noch gefunden werden \cite{Graffi2007}.

\nocite{*}
\bibliographystyle{IEEEtran}
\bibliography{IEEEabrv,bibliothek}

\end{document}

\documentclass[10pt]{IEEEtran}
\usepackage{hyperref}
\usepackage[latin1]{inputenc}

\title{Staukontrolle durch Active Queue Management}
\author{Thomas Fischer, Dominik Billing}

\begin{document}
\maketitle



\section{Einf�hrung und Motivation}
\begin{itemize}
\item Problem

Das Internet hat mittlerweile eine Gr��e erreicht, bei der man sich nicht mehr komplett auf Ende-zu-Ende Staukontrolle verlassen kann \cite{Floyd1997}. Sehr viele Router stellen aufgrund der sternf�rmigen Struktur des Internets Flaschenh�lse dar.
\item Problemstellung

Die mittlere Pufferauslastung soll gering gehalten werden, um E2E Staukontrolle zu erm�glich \cite{Le2003}.
\end{itemize}



\section{Staukontrolle in Netzen}
\begin{itemize}
\item Vorschl�ge zur Staukontrolle

Vorschl�ge zur Gew�hrleistung und Verbesserung der Internetperformance \cite{Braden1998}
\item congestion collapse

Congestion collapse und wie es dazu f�hrt \cite{Nagle1984}.
\item Warum ist das Problem so schwer zu identifizieren

Was sind die Gr�nde f�r das Problem \cite{Jain1990}?
\item Mechanismen zur Staukontrolle in ATM Netzwerken

Auswahlkriterien zwischen den beiden Ans�tzen rate-based und credit-based \cite{Jain1996}.
\item Standard TCP Verhalten bei Staus

Warum ist es keine gute Idee TCP die Staukontrolle selbst zu machen \cite{Morris1997}? Gleichbehandlung aller Datenstr�me \cite{Suter1998}.
\item Explicit Congestion Notification

Vor- und Nachteile von ECN bei TCP \cite{Floyd1994}
\item Router Mechanismen zur Staukontrolle

Vor- und Nachteile von normaler Staukontrolle in Routern \cite{Floyd1997}
\item �berleitung zu Active Queue Management

Active Queue Management ist eine L�sungsansatz zur Staukontrolle \cite{Graffi2007}
\end{itemize}



\section{Definition und Anwendung von Active Queue Management}
\begin{itemize}
\item Effizientes Active Queue Management in Internet Routern \cite{Suter1998a}
\item Dimensionierung von Router Puffern \cite{Appenzeller2004}
\item Stochastische Modellierung und die Theorie von Queues \cite{Wolff1998}
\item Analyse und Simulation eines gleichbehandelnden Queue Algorithmus \cite{Demers1989}
\end{itemize}



\section{Die g�ngigsten Active Queue Management Algorithmen}
\begin{itemize}
\item RED (Random Early Detection) \cite{Floyd1993} \cite{Firoiu2000}

Adaptive RED\cite{Floyd2001}
\item BLUE \cite{Feng2002}
\item ECN \cite{Ramakrishnan2001}
\item PI Controller \cite{Hollot2001}
\end{itemize}



\section{Vergleich der vorgestellten Algorithmen}
\begin{itemize}
\item Vergleich RED, ARED, PI \cite{Le2003}
\item Vergleich RED, PI \cite{Kunniyur2001}
\item Vergleich RED, BLUE, ARED, ECN, PI \cite{Graffi2007}
\end{itemize}



\section{Ausblick und andere Ans�tze}
\begin{itemize}
\item Ein wirklich optimaler Algorithmus muss noch gefunden werden \cite{Graffi2007}
\item Statt Staukontrolle andere Wege suchen (CHOKe) \cite{Pan2000}
\end{itemize}

\nocite{*}
\bibliographystyle{IEEEtran}
\bibliography{IEEEabrv,bibliothek}

\end{document}
